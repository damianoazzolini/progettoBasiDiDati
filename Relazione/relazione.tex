\documentclass[paper=a4, fontsize=11pt,x11names]{report}
\usepackage[T1]{fontenc}
\usepackage{fourier}

%margine a mano
\usepackage[margin=1.1in]{geometry}

\usepackage[utf8]{inputenc}
\usepackage[italian]{babel}
\usepackage[protrusion=true,expansion=true]{microtype}	
%\usepackage{amsmath,amsfonts,amsthm} % Math packages
\usepackage[pdftex]{graphicx}	
\usepackage{url}
\usepackage{graphicx}


%per il grafico
\usepackage{tikz}
\usetikzlibrary{mindmap,trees}
\usepackage{pgfplots} %per istorgrammi

%%% Custom sectioning
\usepackage{sectsty}
\allsectionsfont{\centering \normalfont\scshape}


%%% Custom headers/footers (fancyhdr package)
\usepackage{fancyhdr}
\pagestyle{fancyplain}
\fancyhead{}											% No page header
\fancyfoot[L]{}											% Empty 
\fancyfoot[C]{}											% Empty
\fancyfoot[R]{\thepage}									% Pagenumbering
\renewcommand{\headrulewidth}{0pt}			% Remove header underlines
\renewcommand{\footrulewidth}{0pt}				% Remove footer underlines
\setlength{\headheight}{13.6pt}

%\setlength{\textwidth}{6.5in}

%%% Equation and float numbering
%\numberwithin{equation}{section}		% Equationnumbering: section.eq#
%\numberwithin{figure}{section}			% Figurenumbering: section.fig#
%\numberwithin{table}{section}				% Tablenumbering: section.tab#




%%%%%%%%%%%%% per ER

\usepackage{tikz} 
\usetikzlibrary{er} 
\usetikzlibrary{arrows,automata}
\usetikzlibrary{positioning}
\tikzset{multi attribute/.style={attribute,double distance=1.5pt}} 
\tikzset{derived attribute/.style={attribute,dashed}} 
\tikzset{total/.style={double distance=1.5pt}} 
\tikzset{every entity/.style={draw=orange, fill=orange!20}} 
\tikzset{every attribute/.style={draw=MediumPurple1, fill=MediumPurple1!20}} 
\tikzset{every relationship/.style={draw=Chartreuse2, fill=Chartreuse2!20}} 
\newcommand{\key}[1]{\underline{#1}}

%%%%%%%%%%%%%%%%%


%%% Maketitle metadata
\newcommand{\horrule}[1]{\rule{\linewidth}{#1}} 	% Horizontal rule


\title{
		%\vspace{-1in} 	
		\usefont{OT1}{bch}{b}{n}
		\normalfont \normalsize \textsc{Universit\`a degli Studi di Ferrara \\ Ingegneria Informatica e dell'Automazione
			\\ Basi di Dati} \\ [25pt]
		\horrule{0.5pt} \\[0.4cm]
		\Huge Realizzazione Database per Social Network \date{}\\%[0.2cm]
		\horrule{0.5pt} \\[0.4cm]
		\LARGE Azzolini Damiano - Bertagnon Alessandro \\ [0.4cm]
		%\normalfont \normalsize Azzolini Damiano, Bertasi Francesco, Decarlo Dario, \\
		%\normalfont \normalsize Fazzi Mattia, Fiorini Giovanni, Fontana Niccol\`o, Rambaldi Marco\\[0.4cm]
		\horrule{2pt} \\[0.8cm]
		\includegraphics{logoUnife}
		%\horrule{2pt} \\[0.5cm]
		%%%%%%%%%%%%%%%%%%%%%%%%%%%%%%% MANCA DATA %%%%%%%%%%%%%%%
}


%%% Begin document
\begin{document}
\maketitle

\newpage

\tableofcontents
\thispagestyle{empty}

\listoffigures
\thispagestyle{empty}


\newpage

\pagenumbering{arabic}
%\begin{abstract}
%Your abstract goes here...
%\end{abstract}

%\chapter{Introduction} %solo per report
%This chapter's content...


%%%%%%%%%%%%%%%%%%%%%%%%%%%%%%%%
%%%%%%%%%%%%%%%           \`u

%%%%%%%%%%%%%%%%%%%%%%%%%%% DESCRIZIONE MINIMONDO
\chapter{Minimondo}
\section{Descrizione}
Il progetto di basa sullo sviluppo di un \textit{social network}. Ciascun utente si può iscrivere al social network
fornendo Nome, Cognome, e-mail e password e data di nascita. Al seguito dell'iscrizione, un utente può chiedere l'"amicizia" ad 
un altro utente, pubblicare un
post, che può anche contenere media, che può essere commentato da altri utenti (solamente quelli tra gli amici dell'utente creatore del post). E possibile inoltre inserire, oltre a commenti, delle reazioni ai post. 
La scrittura di un commento o l'applicazione di una reazione ad un post viene segnalata con una notifica all'utente creatore del post, così come la richiesta di amicizia. Ciascuna entità utente, post, commento è caratterizzata da un campo attivo (flag booleano) che indica se, rispettivamente, l'utente si è cancellato dal social network (in questo caso attivo sarà fissato a 0), il post è stato cancellato o il commento è stato rimosso. I post che vengono cancellati dal social network rimangono comunque salvati nel database ma non visualizzati nella home.

\section{Entità}
Di seguito vengono analizzate tutte le entità presenti nel database:

\subsection{Utente}
\centering
\vspace{0.5cm}
\begin{tikzpicture}[
	%->, frecce orietate
	%>=stealth',
  %	%shorten >=1pt,
  	auto,	
  	semithick, %linee più spesse
	node distance=7em,
	scale=0.8, transform shape] 
	
	\node[entity] (utente) {Utente};
	
	\node[attribute] (nome) [above of = utente] {Nome} edge (utente);
	\node[attribute] (email) [below of = utente] {Email} edge (utente);
	\node[attribute] (cognome) [right of = utente] {Cognome} edge (utente); 
	\node[attribute] (datanascite) [below right of = utente] {DataNascita} edge (utente);	
	\node[attribute] (password) [below left of = utente] {Password} edge (utente);
	\node[attribute] (userid) [left of = utente] {\key{UserID}} edge (utente);
	\node[attribute] (attivo) [above right of =  utente] {Attivo} edge (utente);
	
\end{tikzpicture}
\vspace{0.5cm}

%L'entità utente è l'entità principale di tutto il database ed è costituita dai seguenti attributi:
\begin{itemize}
\item \textit{Nome}: nome dell'utente.
\item \textit{Cognome}: cognome dell'utente.
\item \textit{UserID}: identificativo numerico dell'utente e chiave primaria.
\item \textit{Password}: password dell'account.
\item \textit{Email}: email dell'utente. Nel databse non possono esistere due utenti diversi con la stessa mail.
\item \textit{DataNascita}: data di nascita dell'utente.
\item \textit{Attivo}: flag booleano, solitamente fissato a 0, 1 se l'utente si è cancellato dal social network. 
\end{itemize}

\subsection{Amicizia}
\centering
\vspace{0.5cm}
\begin{tikzpicture}[
	%->, frecce orietate
	%>=stealth',
  %	%shorten >=1pt,
  	auto,	
  	semithick, %linee più spesse
	node distance=7em,
	scale=0.8, transform shape] 
	
	\node[entity] (amicizia) {Amicizia}; %entità debole?
	
	\node[attribute] (userid1) [above of = amicizia] {UserID1} edge (amicizia); %da rimuovere
	\node[attribute] (userid2) [above left of = amicizia] {UserID2} edge (amicizia); %da rimuovere
	\node[attribute] (data) [below right of = amicizia] {Data} edge (amicizia);	
	\node[attribute] (ora) [below left of = amicizia] {Ora} edge (amicizia);
	\node[attribute] (stato) [above right of = amicizia] {Stato} edge (amicizia);
\end{tikzpicture}
\vspace{0.5cm}

\begin{itemize}
\item \textit{UserID1}: rappresenta l'identificativo dell'utente che richiede l'amicizia, chiave esterna.
\item \textit{UserID2}: rappresenta l'identificativo dell'utente che riceve la richiesta di amicizia, chiave esterna.
\item \textit{Data}: data della richiesta dell'amicizia.
\item \textit{Ora}: ora della richiesta dell'amicizia.
\item \textit{Stato}: variabile intera che rappresenta se l'amicizia è stata richiesta, accettata, in sospeso o rifiutata.
\end{itemize}

\subsection{Post}
\centering
\vspace{0.5cm}
\begin{tikzpicture}[
	%->, frecce orietate
	%>=stealth',
  %	%shorten >=1pt,
  	auto,	
  	semithick, %linee più spesse
	node distance=7em,
	scale=0.8, transform shape] 
	
	\node[entity] (post) {Post};
	
	\node[attribute] (postid) [below left of=post] {\key{PostID}} edge (post);
	\node[attribute] (contenuto) [above left of=post] {Contenuto} edge (post);
	\node[attribute] (data) [below right of=post] {Data} edge (post);
	\node[attribute] (ora) [right of = post] {Ora} edge (post);
	\node[attribute] (userid) [above right of=post] {UserID} edge (post); %da rimuovere
	\node[attribute] (attivo) [left of = post] {Attivo} edge (post);
	
	
\end{tikzpicture}
\vspace{0.5cm}

\begin{itemize}
\item \textit{PostID1}: rappresenta l'identificativo del post, funge da chiave primaria.
\item \textit{Contenuto}: campo di testo che rappresenta il contenuto del post.
\item \textit{Data}: data di pubblicazione del post.
\item \textit{Ora}: ora di pubblicazione del post.
\item \textit{UserID}: chiave esterna, identificativo dell'utente autore del post.
\item \textit{Attivo}: flag booleano che indica se il post è attivo (1) o cancellato (0).
\end{itemize}

\subsection{Commento}
\centering
\vspace{0.5cm}
\begin{tikzpicture}[
	%->, frecce orietate
	%>=stealth',
  %	%shorten >=1pt,
  	auto,	
  	semithick, %linee più spesse
	node distance=7em,
	scale=0.8, transform shape] 

	\node[entity] (commento) {Commento}; %entità debole post
	
	\node[attribute] (commentid) [above right of = commento] {\key{CommentoID}} edge (commento);
	\node[attribute] (contenuto) [above left of = commento] {Contenuto} edge (commento);
	\node[attribute] (userid) [left of = commento] {UserID} edge (commento);	 %da rimuovere
	\node[attribute] (data) [below left of = commento] {Data} edge (commento); 
	\node[attribute] (postid) [below of = commento] {PostID} edge (commento);
	\node[attribute] (ora) [below right of = commento] {Ora} edge (commento); %da rimuovere
	\node[attribute] (attivo) [right of = commento] {Attivo} edge (commento);
		
\end{tikzpicture}
\vspace{0.5cm}

\begin{itemize}
\item \textit{CommentoID}: identificativo numerico del commento e chiave primaria.
\item \textit{Contenuto}: contenuto del commento.
\item \textit{UserID}: identificativo dell'utente autore del commento, chiave esterna.
\item \textit{Data}: data del commento.
\item \textit{Ora}: ora del commento.
\item \textit{PostID}: identificativo numerico del post commentato, chiave esterna.
\item \textit{Attivo}: flag booleano che indica se il commento è attivo (1) o cancellato (0).
\end{itemize}

\subsection{Media}
\centering
\vspace{0.5cm}
\begin{tikzpicture}[
	%->, frecce orietate
	%>=stealth',
  %	%shorten >=1pt,
  	auto,	
  	semithick, %linee più spesse
	node distance=7em,
	scale=0.8, transform shape] 

	\node[entity] (media) {Media}; %entia debole post
	
	\node[attribute] (mediaid) [above right of=media] {\key{MediaID}} edge (media);
	\node[attribute] (percorso) [above left of=media] {Percorso} edge (media);
	\node[attribute] (postid) [below of=media] {PostID} edge (media); %modificare
		
\end{tikzpicture}
\vspace{0.5cm}

\begin{itemize}
\item \textit{MediaID}: identificativo del media (foto, video, ...), chiave primaria.
\item \textit{Percorso}: link al media.
\item \textit{PostID}: identificativo del post nel quale il media viene inserito.
\end{itemize}


\subsection{Reazione}

\centering
\vspace{0.5cm}
\begin{tikzpicture}[
	%->, frecce orietate
	%>=stealth',
  %	%shorten >=1pt,
  	auto,	
  	semithick, %linee più spesse
	node distance=7em,
	scale=0.8, transform shape] 
	
	\node[entity] (reazione) {Reazione}; %entita debole post
	
	\node[attribute] (userid) [above right of = reazione] {UserID} edge (reazione);
	\node[attribute] (postid) [below left of = reazione] {PostID} edge (reazione); %modificare
	\node[attribute] (flag) [below right of = reazione] {Attivo} edge (reazione);
	\node[attribute] (userid) [above left of = reazione] {\key{ReazioneID}} edge (reazione);

\end{tikzpicture}
\vspace{0.5cm}

\begin{itemize}
\item \textit{ReazioneID}: identificativo numerico della reazione.
\item \textit{UserID}: identificativo dell'utente che mette la reazione.
\item \textit{PostID}: identificativo del post sul quale viene messa una reazione.
\item \textit{Attivo}: flag booleano che indica se ila reazione è attiva (1) o è stata cancellata (0).
\end{itemize}

\subsection{Notifica}

\centering
\vspace{0.5cm}
\begin{tikzpicture}[
	%->, frecce orietate
	%>=stealth',
  %	%shorten >=1pt,
  	auto,	
  	semithick, %linee più spesse
	node distance=7em,
	scale=0.8, transform shape] 
	
	\node[entity] (notifica) {Notifica};
	
	\node[attribute] (userid) [above left of=notifica] {UserID} edge (notifica); %modificare
	\node[attribute] (tipo) [above of=notifica] {Tipo} edge (notifica);
	\node[attribute] (tipo_id) [above right of=notifica] {TipoID} edge (notifica);
	\node[attribute] (id) [below left of = notifica] {\key{NotificaID}} edge (notifica); 	
	\node[attribute] (letta) [below right of=notifica] {Letta} edge (notifica);
	
\end{tikzpicture}
\vspace{0.5cm}

\begin{itemize}
\item \textit{NotiificaID}: identificativo numerico della notifica.
\item \textit{Tipo}:
\item \textit{TipoID}:
\item \textit{UserID}:
\item \textit{Letta}:
\end{itemize}

\subsection{Gruppo}
Entità da aggiungere



\section{Relazioni}
%UNA SUBSECTION PER OGNI RELAZIONE?
%Le relazioni che collegano le entità sono le seguenti:

%genera riceve accetta scrive scatena richiede lancia possiede crea valuta mette contiene

%riceve
\begin{figure}[h!]
\centering
\begin{tikzpicture}[auto, semithick, scale=1, transform shape] 
	
	\node[entity] (notifica) {Notifica};
	\node[relationship] (riceve) [right = 2cm of notifica] {Riceve} edge node {N} (notifica);
	\node[entity] (utente) [right = 2cm of riceve] {Utente} edge node {1} (riceve);
\end{tikzpicture}
\caption{Un utente riceve N notifiche ma una determinata notifica è ricevuta da un solo utente.}
\end{figure}

%richiede
\begin{figure}[h!]
\centering
\begin{tikzpicture}[auto, semithick, scale=1, transform shape] 
	
	\node[entity] (amicizia) {Amicizia};
	\node[relationship] (richiede) [right = 2cm of amicizia] {Richiede} edge node {N} (amicizia);
	\node[entity] (utente) [right = 2cm of riceve] {Utente} edge node {1} (richiede);
\end{tikzpicture}
\caption{Un utente riceve N richieste di amicizia ma una determinata richiesta è ricevuta da un solo utente.}
\end{figure}

%accetta
\begin{figure}[h!]
\centering
\begin{tikzpicture}[auto, semithick, scale=1, transform shape] 
	
	\node[entity] (amicizia) {Amicizia};
	\node[relationship] (accetta) [right = 2cm of amicizia] {Accetta} edge node {N} (amicizia);
	\node[entity] (utente) [right = 2cm of riceve] {Utente} edge node {1} (accetta);
\end{tikzpicture}
\caption{Un utente accetta N richieste di amicizia ma una determinata richiesta è accettata da un solo utente.}
\end{figure}

%scatena
\begin{figure}[h!]
\centering
\begin{tikzpicture}[auto, semithick, scale=1, transform shape] 
	
	\node[entity] (notifica) {Notifica};
	\node[relationship] (scatena) [right = 2cm of notifica] {Scatena} edge node {1} (notifica);
	\node[entity] (Reazione) [right = 2cm of scatena] {Reazione} edge node {1} (scatena);
\end{tikzpicture}
\caption{Una reazione scatena una notifica e una determinata notifica è scatenata da una reazione.}
\end{figure}

%genera
\begin{figure}[h!]
\centering
\begin{tikzpicture}[auto, semithick, scale=1, transform shape] 
	
	\node[entity] (amicizia) {Amicizia};
	\node[relationship] (genera) [right = 2cm of amicizia] {Genera} edge node {N} (amicizia);
	\node[entity] (notifica) [right = 2cm of riceve] {Notifica} edge node {1} (genera);
\end{tikzpicture}
\caption{Una richiesta di amicizia genera una notifica e una determinata notifica e generata da una richiesta.}
\end{figure}

%scrive
\begin{figure}[h!]
\centering
\begin{tikzpicture}[auto, semithick, scale=1, transform shape] 
	
	\node[entity] (utente) {Utente};
	\node[relationship] (scrive) [right = 2cm of utente] {Scrive} edge node {1} (utente);
	\node[entity] (commento) [right = 2cm of riceve] {Commento} edge node {N} (scrive);
\end{tikzpicture}
\caption{Un utente scrive n commenti ma un determinato commento può essere scritto solamente da un utente.}
\end{figure}

%lancia
\begin{figure}[h!]
\centering
\begin{tikzpicture}[auto, semithick, scale=1, transform shape] 
	
	\node[entity] (commento) {Commento};
	\node[relationship] (lancia) [right = 2cm of commento] {Lancia} edge node {1} (commento);
	\node[entity] (notifica) [right = 2cm of lancia] {Notifica} edge node {1} (lancia);
\end{tikzpicture}
\caption{Un commento lancia una notifica e una determinata notifica può essere lanciata da un solo commento.}
\end{figure}

%possiede
\begin{figure}[h!]
\centering
\begin{tikzpicture}[auto, semithick, scale=1, transform shape] 
	
	\node[entity] (commento) {Commento};
	\node[relationship] (possiede) [right = 2cm of commento] {Possiede} edge node {N} (commento);
	\node[entity] (post) [right = 2cm of possiede] {Post} edge node {1} (possiede);
\end{tikzpicture}
\caption{Un determinato commento può essere fatto solamente su un post ma un post può contenere n commenti.}
\end{figure}

%crea
\begin{figure}[h!]
\centering
\begin{tikzpicture}[auto, semithick, scale=1, transform shape] 
	
	\node[entity] (utente) {Utente};
	\node[relationship] (crea) [right = 2cm of utente] {Crea} edge node {1} (utente);
	\node[entity] (post) [right = 2cm of crea] {Post} edge node {N} (crea);
\end{tikzpicture}
\caption{Un utente crea n post ma un determinato post è scritto solamente da un utente.}
\end{figure}

%valuta
\begin{figure}[h!]
\centering
\begin{tikzpicture}[auto, semithick, scale=1, transform shape] 
	
	\node[entity] (reazione) {Reazione};
	\node[relationship] (valuta) [right = 2cm of reazione] {Valuta} edge node {N} (reazione);
	\node[entity] (post) [right = 2cm of crea] {Post} edge node {1} (valuta);
\end{tikzpicture}
\caption{Una reazione valuta un solo post ma un post può essere valutato da più reazioni.}
\end{figure}

%mette
\begin{figure}[h!]
\centering
\begin{tikzpicture}[auto, semithick, scale=1, transform shape] 
	
	\node[entity] (utente) {Utente};
	\node[relationship] (mette) [right = 2cm of utente] {Mette} edge node {1} (utente);
	\node[entity] (reazione) [right = 2cm of mette] {Reazione} edge node {N} (mette);
\end{tikzpicture}
\caption{Un utente può mettere n reazioni ma una reazione può essere messa da un solo utente.}
\end{figure}

%contiene
\begin{figure}[h!]
\centering
\begin{tikzpicture}[auto, semithick, scale=1, transform shape] 
	
	\node[entity] (post) {Post};
	\node[relationship] (contiene) [right = 2cm of post] {Contiene} edge node {1} (post);
	\node[entity] (media) [right = 2cm of contiene] {Media} edge node {N} (contiene);
\end{tikzpicture}
\caption{Un post contiene n media ma un media può essere in un solo post.}
\end{figure}

%eventuali nuovi

\clearpage
\section{Schema ER Completo}
Da inserire in una pagina nuova

%%%%%%%%%%%%%%%%%%%%%%%%%%% Da Modello ER a Modello Relazionale
\chapter{Da Modello ER a Modello Relazionale}

%%%%%%%%%%%%%%%%%%%%%%%%%%% NORMALIZZAZIONE
\chapter{Normalizzazione}

%%%%%%%%%%%%%%%%%%%%%%%%%%% CODICE SQL
\chapter{Codice SQL}

%%%%%%%%%%%%%%%%%%%%%%%%%%% INTERROGAZIONI
\chapter{Query}

%%%%%%%%%%%%%%%%%%%%%%%%%%% INTERFACCIA
\chapter{Interfaccia}

%%%%%%%%%%%%%%%%%%%%%%%%% BIBLIOGRAFIA


\begin{thebibliography}{9}


\bibitem{leibold}
 Marius Leibold, Gilbert J. B. Probst, Michael Gibbert,
  \textit{Strategic Management in the Knowledge Economy: New Approaches and Business Applications},
  John Wiley \& Sons,
 2007.
 
\bibitem{cartonChina}
	Tetra Pak launches FSC cartons in China, \textit{http://beta.nepcon.org/newsroom/tetra-pak-launches-fsc-cartons-china},
	10 Giugno 2010.
	
\bibitem{circular}
	Circular Economy, Sustainable Materials Management, and the Importance of KPIs: \textit{https://sustainablepackaging.org/circular-economy-sustainable-materials-management-importance-kpis/},
	17 Maggio 2017.

\end{thebibliography}


%%% End document
\end{document}