\documentclass[paper=a4, fontsize=11pt,x11names]{report}
\usepackage[T1]{fontenc}
\usepackage{fourier}

%margine a mano
\usepackage[margin=1.1in]{geometry}

\usepackage[utf8]{inputenc}
\usepackage[italian]{babel}
\usepackage[protrusion=true,expansion=true]{microtype}	
%\usepackage{amsmath,amsfonts,amsthm} % Math packages
\usepackage[pdftex]{graphicx}	
\usepackage{url}
\usepackage{graphicx}


%per il grafico
\usepackage{tikz}
\usetikzlibrary{mindmap,trees}
\usepackage{pgfplots} %per istorgrammi

\usepackage{float}

\usepackage{pdflscape}

%%% Custom sectioning
\usepackage{sectsty}
\allsectionsfont{\centering \normalfont\scshape}

%%% Custom headers/footers (fancyhdr package)
\usepackage{fancyhdr}
\pagestyle{fancyplain}
\fancyhead{}											% No page header
\fancyfoot[L]{}											% Empty 
\fancyfoot[C]{}											% Empty
\fancyfoot[R]{\thepage}									% Pagenumbering
\renewcommand{\headrulewidth}{0pt}			% Remove header underlines
\renewcommand{\footrulewidth}{0pt}				% Remove footer underlines
\setlength{\headheight}{13.6pt}

%\setlength{\textwidth}{6.5in}

%%% Equation and float numbering
%\numberwithin{equation}{section}		% Equationnumbering: section.eq#
%\numberwithin{figure}{section}			% Figurenumbering: section.fig#
%\numberwithin{table}{section}				% Tablenumbering: section.tab#




%%%%%%%%%%%%% per ER

\usepackage{tikz} 
\usetikzlibrary{er} 
\usetikzlibrary{arrows,automata}
\usetikzlibrary{positioning}
\tikzset{multi attribute/.style={attribute,double distance=1.5pt}} 
\tikzset{derived attribute/.style={attribute,dashed}} 
\tikzset{total/.style={double distance=1.5pt}} 
\tikzset{every entity/.style={draw=orange, fill=orange!20}} 
\tikzset{every attribute/.style={draw=MediumPurple1, fill=MediumPurple1!20}} 
\tikzset{every relationship/.style={draw=Chartreuse2, fill=Chartreuse2!20}} 
\newcommand{\key}[1]{\underline{#1}}

%%%%%%%%%%%%%%%%%


%%% Maketitle metadata
\newcommand{\horrule}[1]{\rule{\linewidth}{#1}} 	% Horizontal rule


\title{
		%\vspace{-1in} 	
		\usefont{OT1}{bch}{b}{n}
		\normalfont \normalsize \textsc{Universit\`a degli Studi di Ferrara \\ Ingegneria Informatica e dell'Automazione
			\\ Basi di Dati} \\ [25pt]
		\horrule{0.5pt} \\[0.4cm]
		\Huge Realizzazione Database per Ospedale \date{}\\%[0.2cm]
		\horrule{0.5pt} \\[0.4cm]
		\LARGE Azzolini Damiano - Bertagnon Alessandro \\ [0.4cm]
		%\normalfont \normalsize Azzolini Damiano, Bertasi Francesco, Decarlo Dario, \\
		%\normalfont \normalsize Fazzi Mattia, Fiorini Giovanni, Fontana Niccol\`o, Rambaldi Marco\\[0.4cm]
		\horrule{2pt} \\[0.8cm]
		\includegraphics{logoUnife}
		%\horrule{2pt} \\[0.5cm]
		%%%%%%%%%%%%%%%%%%%%%%%%%%%%%%% MANCA DATA %%%%%%%%%%%%%%%
}


%%% Begin document
\begin{document}
\maketitle

\newpage

\tableofcontents
\thispagestyle{empty}

\listoffigures
\thispagestyle{empty}


\newpage

\pagenumbering{arabic}
%\begin{abstract}
%Your abstract goes here...
%\end{abstract}

%\chapter{Introduction} %solo per report
%This chapter's content...


%%%%%%%%%%%%%%%%%%%%%%%%%%%%%%%%
%%%%%%%%%%%%%%%           \`u

%%%%%%%%%%%%%%%%%%%%%%%%%%% DESCRIZIONE MINIMONDO
\chapter{Minimondo}
\section{Descrizione}
Il progetto di basa sulla realizzazione di una applicazione web per la gestione di una clinica privata. Alla piattaforma possono accedere 5 tipi di utente:
\begin{itemize}
\item paziente
\item medico
\item infermiere
\item impiegato
\item amministratore
\end{itemize}  

La clinica in questione eroga diversi tipi di \textbf{prestazioni} ai suoi utenti, ad esempio: visite specialistiche, esami diagnostici, day surgery e terapie. Ogni prestazione può essere effettuata da uno o più membri dello \textbf{staff} (a seconda della complessità) in una delle \textbf{sale} della clinica. Al termine di ogni prestazione il medico compila un \textbf{referto} corrispondete alla prestazione appena effettuata. Il sistema deve anche gestire i \textbf{farmaci} assunti dagli utenti e utilizzati durante le prestazioni. Per motivi di organizzazione interna ogni membro del personale e ogni sala afferisce a uno specifico \textbf{reparto} della clinica. Più in dettaglio:\\

L'utente \textbf{PAZIENTE} potrà:
\begin{itemize}
\item Registrarsi sulla piattaforma, fare il login e modificare il proprio profilo.
\item Aggiungere/Rimuovere i farmaci che assume regolarmente.
\item Visionare le prestazioni effettuate con i referti corrispondenti.
\end{itemize}

L'utente \textbf{MEDICO} potrà:
\begin{itemize}
\item Fare il login sulla piattaforma e visionare il proprio profilo.
\item Visionare le schede personali dei pazienti (compresi i farmaci assunti).
\item Visionare le prestazioni e i relativi referti.
\item Aggiungere/Modificare/Cancellare i referti delle prestazioni a cui ha preso parte.
\item Aggiungere/Rimuovere i farmaci utilizzati nelle prestazioni a cui ha preso parte.
\item Aggiungere personale alle prestazioni che gli sono state assegnate.
\end{itemize}

L'utente \textbf{INFERMIERE} potrà:
\begin{itemize}
\item Fare il login sulla piattaforma e visionare il proprio profilo.
\item Visionare le schede personali dei pazienti (compresi i farmaci assunti).
\item Visionare le prestazioni assegnate.
\item Aggiungere/Rimuovere i farmaci utilizzati nelle prestazioni alle quali ha preso parte.
\end{itemize}

L'utente \textbf{IMPIEGATO} potrà:
\begin{itemize}
\item Fare il login sulla piattaforma e visionare il proprio profilo.
\item Prenotare le prestazioni per i pazienti associando ad esse i medici che dovranno effettuarle.
\item Visualizzare lo storico delle prestazioni effettuate dai pazienti (ma non i referti). 
\item Gestire il personale:
	\begin{itemize}
		\item Modificare lo stipendio dei vari membri dello staff.
		\item Modificare il reparto di appartenenza.
	\end{itemize}
	
\item Aggiungere/Modificare/Cancellare le tipologie di prestazioni. 
\item Aggiungere/Modificare/Cancellare i farmaci nella lista della farmacia.
\end{itemize}

L'utente \textbf{AMMINISTRATORE} potrà:
\begin{itemize}
\item Fare il login sulla piattaforma e visionare il proprio profilo.
\item Fare tutto quello che fanno gli utenti precedenti.
\item Aggiungere/Modificare/Cancellare gli utenti Staff della clinica.
\item Aggiungere/Modificare/Cancellare le sale della clinica.
\item Aggiungere/Modificare/Cancellare i reparti della clinica.
\item Gestire tutta la base utenti.
\item Aggiungere/Rimuovere i singoli ruoli (permessi) agli utenti.
\end{itemize}

\section{Entità}
Di seguito vengono analizzate tutte le entità presenti nel database:


%%%%%%%%%%%%%%%%%%%%%%%%%%%%%%%%%%%%%%
% FINIRE DI MODIFICARE I CAMPI
%%%%%%%%%%%%%%%%%%%%%%%%%%%%%%%%



\subsection{Utente}
\begin{center}
\vspace{0.5cm}
\begin{tikzpicture}[
	%->, frecce orietate
	%>=stealth',
  %	%shorten >=1pt,
  	auto,	
  	semithick, %linee più spesse
	node distance=7em,
	scale=0.8, transform shape] 
	
	\node[entity] (utente) {Utente};
	
	\node[attribute] (id) [above of = utente] {\key{ID}} edge (utente);
	\node[attribute] (nome) [above right of = utente] {Nome} edge (utente);
	\node[attribute] (cognome) [right of = nome] {Cognome} edge (utente);
	\node[attribute] (dataNascita) [below of = utente] {DataNascita} edge (utente);
	\node[attribute] (sesso) [below right of = utente] {Sesso} edge (utente);
	\node[attribute] (cf) [right of = sesso] {C.F.} edge (utente);
	\node[attribute] (email) [below left of = utente] {Email} edge (utente);
	\node[attribute] (pw) [left of = email] {Password} edge (utente);
	\node[attribute] (telefono) [left of = utente] {Telefono} edge (utente);
	\node[attribute] (attivo) [right of = utente] {Attivo} edge (utente);
	
	\node[attribute] (indirizzo) [above left of = utente] {Indirizzo} edge (utente);
	\node[attribute] (stato) [above of = indirizzo] {Stato} edge (indirizzo);
	\node[attribute] (comune) [left of = stato] {Comune} edge (indirizzo);
	\node[attribute] (via) [left of = indirizzo] {Via} edge (indirizzo);
	\node[attribute] (numeroCivico) [right of = stato] {Civico} edge (indirizzo);
	\node[attribute] (provincia) [left of = comune] {Provincia} edge (indirizzo);	
\end{tikzpicture}
\vspace{0.5cm}
\end{center}
L'entità \textit{Utente} contiene tutte le informazioni riguardo alle persone che usufruiscono di un servizio ospedaliero, sia che siano componenti dello staff che pazienti. Ogni utente è caratterizzato in maniera univoca da un \textit{ID}. Il contenuto degli attributi \textit{email} e \textit{C.F.} (codice fiscale) deve essere unico (non possono esserci due utenti con lo stesso valore nel campo \textit{mail} e/o \textit{C.F.}). L'attributo \textit{Attivo} viene utilizzato per indicare se un utente è attivo oppure no (l'utente ha la possibilità di cancellarsi dall'ospedale, tuttavia non viene eliminata la riga corrispondente dal database, ma viene impostato a 0 l'attributo \textit{attivo}).

\subsection{Paziente}
\begin{center}
\vspace{0.5cm}
\begin{tikzpicture}[
	%->, frecce orietate
	%>=stealth',
  %	%shorten >=1pt,
  	auto,	
  	semithick, %linee più spesse
	node distance=7em,
	scale=0.8, transform shape] 
	
	\node[entity] (paziente) {Paziente};
	
	\node[attribute] (altezza) [above left of = paziente] {Altezza} edge (paziente);
	\node[attribute] (note) [left of = paziente] {Note} edge (utente);
	\node[attribute] (peso) [right of = paziente] {Peso} edge (paziente);
	\node[attribute] (timestamp) [above right of = paziente] {Timestamp} edge (paziente);		
\end{tikzpicture}
\vspace{0.5cm}
\end{center}
L'entità \textit{Paziente} è una specializzazione della entità \textit{Utente}. Ha come chiave esterna l'\textit{ID} dell'utente al quale si riferisce e presenta anche un attributo \textit{note} per eventuali informazioni aggiuntive.

\subsection{Staff}
\begin{center}
\vspace{0.5cm}
\begin{tikzpicture}[
	%->, frecce orietate
	%>=stealth',
  %	%shorten >=1pt,
  	auto,	
  	semithick, %linee più spesse
	node distance=7em,
	scale=0.8, transform shape] 
	
	\node[entity] (staff) {Staff};
	
	\node[attribute] (altezza) [above of = staff] {Identificativo} edge (staff);
	\node[attribute] (stipendio) [above left = 1cm of staff] {Stipendio} edge (staff);
	\node[attribute] (timestamp) [above right = 1cm of staff] {Timestamp} edge (staff);		
\end{tikzpicture}
\vspace{0.5cm}
\end{center}
L'entità \textit{Staff}, analogamente ad \textit{Paziente}, è una specializzazione di \textit{Utente}. Ha come chiave esterna l'\textit{ID} dell'utente al quale si riferisce e il reparto di appartenenza. Inoltre presenta un attributo \textit{Identificativo} per specificarne la funzione.

\subsection{Utente - Paziente - Staff}
\begin{center}
\vspace{0.5cm}
\begin{tikzpicture}[
	%->, frecce orietate
	%>=stealth',
  %	%shorten >=1pt,
  	auto,	
  	semithick, %linee più spesse
	node distance=7em,
	scale=0.8, transform shape] 
	
	\node[entity] (utente) {Utente};
	\node[relationship] (isa) [below of = utente] {ISA} edge (utente);	
	\node[entity] (staff) [below left of = isa] {Staff} edge (isa);
	\node[entity] (paziente) [below right of = isa] {Paziente} edge (isa);
	
\end{tikzpicture}
\vspace{0.5cm}
\end{center}
Le entità \textit{Paziente} e \textit{Staff} sono disgiunte: non può esistere un \textit{Utente} nel database che appartenga sia a \textit{Paziente} che \textit{Staff}.

\subsection{Reparto}
\begin{center}
\vspace{0.5cm}
\begin{tikzpicture}[
	%->, frecce orietate
	%>=stealth',
  %	%shorten >=1pt,
  	auto,	
  	semithick, %linee più spesse
	node distance=7em,
	scale=0.8, transform shape] 
	
	\node[entity] (reparto) {Reparto};
	
	\node[attribute] (id) [above of = reparto] {\key{ID}} edge (reparto);
	\node[attribute] (nome) [above left = 1cm of reparto] {Nome} edge (reparto);
	\node[attribute] (identificativo) [above right = 1cm of reparto] {Identificativo} edge (reparto);
	\node[attribute] (descrizione) [left = 1cm of reparto] {Descrizione} edge (reparto);
	\node[attribute] (timestamp) [right = 1cm of reparto] {Timestamp} edge (reparto);				
\end{tikzpicture}
\vspace{0.5cm}
\end{center}
\textit{Reparto} caratterizza un particolare reparto dell'ospedale (cardiologia, pneumologia, ecc) attraverso l'attributo \textit{Nome}. Tuttavia ogni reparto è caratterizzato anche da un \textit{ID} unico. 

\subsection{Sala}
\begin{center}
\vspace{0.5cm}
\begin{tikzpicture}[
	%->, frecce orietate
	%>=stealth',
  %	%shorten >=1pt,
  	auto,	
  	semithick, %linee più spesse
	node distance=7em,
	scale=0.8, transform shape] 
	
	\node[entity] (sala) {Sala};
	
	\node[attribute] (id) [above of = sala] {\key{ID}} edge (sala);
	\node[attribute] (identificativo) [above left = 1cm of sala] {Identificativo} edge (sala);
	\node[attribute] (descrizione) [above right = 1cm of sala] {Descrizione} edge (sala);
	\node[attribute] (timestamp) [left = 1cm of sala] {Timestamp} edge (sala);	
	\node[attribute] (piano) [right = 1cm of sala] {Piano} edge (sala);				
\end{tikzpicture}
\vspace{0.5cm}
\end{center}
L'entità \textit{Sala} rappresenta le varie sale disponibili nell'ospedale e ha come chiave esterna l'\textit{ID} del reparto alla quale è assegnata. 

\subsection{Farmaco}
\begin{center}
\vspace{0.5cm}
\begin{tikzpicture}[
	%->, frecce orietate
	%>=stealth',
  %	%shorten >=1pt,
  	auto,	
  	semithick, %linee più spesse
	node distance=7em,
	scale=0.8, transform shape] 
	
	\node[entity] (farmaco) {Farmaco};
	
	\node[attribute] (id) [above of = farmaco] {\key{ID}} edge (farmaco);
	\node[attribute] (nome) [above left = 1cm of sala] {Nome} edge (farmaco);
	\node[attribute] (descrizione) [above right = 1cm of farmaco] {Descrizione} edge (farmaco);
	\node[attribute] (timestamp) [left = 1cm of farmaco] {Timestamp} edge (farmaco);	
	\node[attribute] (categoria) [right = 1cm of farmaco] {Categoria} edge (farmaco);				
\end{tikzpicture}
\vspace{0.5cm}
\end{center}
L'entità \textit{Farmaco} rappresenta ciascun farmaco che viene assunto dal paziente (eventualmente anche prima di essere diventato un paziente dell'ospedale). I farmaci possono essere prescritti a seguito di una \textit{Prestazione} attraverso un \textit{Referto}. Ciascun \textit{Farmaco} è caratterizzato da una categoria (salvavita, pressione, ecc) ed include un campo di testo \textit{Descrizione} dove può essere inserita la posologia.

\subsection{Prestazione}
\begin{center}
\vspace{0.5cm}
\begin{tikzpicture}[
	%->, frecce orietate
	%>=stealth',
  %	%shorten >=1pt,
  	auto,	
  	semithick, %linee più spesse
	node distance=7em,
	scale=0.8, transform shape] 
	
	\node[entity] (prestazione) {Prestazione};
	
	\node[attribute] (id) [below right = 1cm of prestazione] {\key{ID}} edge (prestazione);
	\node[attribute] (effettuata) [above left = 1cm of prestazione] {Effettuata} edge (prestazione);
	\node[attribute] (note) [right = 1cm of prestazione] {Note} edge (prestazione);
	\node[attribute] (attivo) [left = 1cm of prestazione] {Attivo} edge (prestazione);	
	\node[attribute] (timestamp) [above right = 1cm of prestazione] {Timestamp} edge (prestazione);	
	\node[attribute] (identificativo) [below left = 1cm of prestazione] {Identificativo} edge (prestazione);			
\end{tikzpicture}
\vspace{0.5cm}
\end{center}
L'entità \textit{Prestazione} rappresenta una prestazione effettuata all'interno dell'ospedale, sia visita medica che operazione chirurgica. Ogni prestazione è caratterizzata da un \textit{ID} univoco e da un \textit{Identificativo} per distinguerne i vari tipi. L'attributo \textit{Attivo} è stato inserito per discriminare le prestazioni prenotate (in questo caso \textit{Attivo} viene posto a 1) e cancellate (\textit{Attivo} a 0). \textit{Effettuata} viene impostato a 1 se la prestazione è stata effettuata, 0 se invece deve essere ancora effettuata: in entrambi i casi, la prestazione non deve essere stata cancellata (\textit{Attivo} deve essere impostato a 1).

\subsection{Referto}
\begin{center}
\vspace{0.5cm}
\begin{tikzpicture}[
	%->, frecce orietate
	%>=stealth',
  %	%shorten >=1pt,
  	auto,	
  	semithick, %linee più spesse
	node distance=7em,
	scale=0.8, transform shape] 
	
	\node[entity] (referto) {Referto};
	
	\node[attribute] (note) [above of = referto] {Note} edge (referto);
	\node[attribute] (esito) [above right = 1cm of farmaco] {Esito} edge (referto);
	\node[attribute] (timestamp) [above left = 1cm of referto] {Timestamp} edge (referto);				
\end{tikzpicture}
\vspace{0.5cm}
\end{center}
L'entità \textit{Referto} è una entità debole, collegata a prestazione. Infatti non ha chiave primaria ed ha come chiavi esterne l'\textit{ID} del paziente e l'\textit{ID} della prestazione alla quale fa riferimento.

%\node[isa] (isa1) [below = of student, node %distance=5em] {ISA} edge node [auto,swap] %{disjoint} (student)

\section{Associazioni}
Le associazioni rappresentano i vari legami che intercorrono tra le varie entità. Sono le seguenti
\begin{itemize}
\item Utilizza: associazione \textit{N:M} tra \textit{Paziente} e \textit{Farmaco}: un \textit{Paziente} può utilizzare diversi farmaci e analogamente un \textit{Farmaco} può essere utilizzato da diversi pazienti.
\item Riguarda: relazione \textit{1:N} tra \textit{Paziente} e \textit{Referto}: un \textit{Paziente} può avere associati ad esso diversi referti ma uno specifico \textit{Referto} è associato ad un solo paziente. Inoltre un referto deve essere necessariamente essere associato ad un paziente, infatti nello schema ER è legato da un vincolo di partecipazione totale.
\item Ottiene: relazione \textit{1:N} tra \textit{Paziente} e \textit{Prestazione}: un \textit{Paziente} può ottenere diverse prestazioni, ma una specifica \textit{Prestazione} deve essere associata univocamente ad un utente, viene quindi rappresentata con un vincolo di partecipazione totale.
\item Effettua: relazione \textit{M:N} tra \textit{Staff} e \textit{Prestazione}: un membro di uno \textit{Staff} può effettuare più prestazioni e ciascuna \textit{Prestazione} può essere effettuata da più membri dello staff (si pensi per esempio ad una operazione chirurgica che coinvolge diversi membri dello staff come anestesista e chirurgo). \textit{Prestazione} ha un vincolo di partecipazione totale.
\item Appartiene: relazione \textit{N:1} tra \textit{Staff} e \textit{Reparto}: un membro dello \textit{Staff} deve appartenere necessariamente (vincolo di partecipazione totale) ad un solo reparto ma un \textit{Reparto} comprende più membri dello staff.
\item Emette: relazione \textit{1:1} tra \textit{Prestazione} e \textit{Referto}: ogni \textit{Prestazione} ha un unico referto ed uno specifico \textit{Referto} è associato ad una sola prestazione. \textit{Referto} è entità debole (non può esistere senza una prestazione) quindi è caratterizzato da un vincolo di partecipazione totale. Essendo una relazione \textit{1 - 1}, si sarebbero potuti includere gli attributi di \textit{Referto} all'interno di \textit{Prestazione}. Tuttavia è stata creata l'entità \textit{Prestazione} per poter semplificare le query e poter risalire, per esempio, in maniera veloce a tutti i referti riguardanti un determinato utente.
\item Prescrive: relazione \textit{N:M} tra \textit{Prestazione} e \textit{Farmaco}: una \textit{Prestazione} può descrivere uno o più farmaci e un \textit{Farmaco} può essere prescritto da una prestazione (ricordiamo che nel database possono essere inseriti anche farmaci non prescritti da una prestazione, che l'utente assumeva già in maniera autonoma).
\item Ha: relazione \textit{N:1} tra \textit{Prestazione} e \textit{Sala}: una \textit{Prestazione} deve (vincolo partecipazione totale) utilizzare una sola sala ma una \textit{Sala} viene utilizzata per più prestazioni.
\item Rep Prest: relazione \textit{N:1} tra \textit{Prestazione} e \textit{Reparto}: una \textit{Prestazione} deve (vincolo di partecipazione totale) essere effettuata in un reparto ma in un \textit{Reparto} possono essere effettuate più prestazioni.
\item Rep Sala: relazione \textit{N:1} tra \textit{Sala} e \textit{Reparto}: una \textit{Sala} deve (vincolo di partecipazione totale) essere assegnata ad un reparto mente ad un \textit{Reparto} sono assegnate più sale. 
\end{itemize}


%\clearpage
%\section{Schema ER Completo}
\begin{landscape}
\begin{figure}[h!]
\includegraphics[scale=0.8]{Diagramma}
\end{figure}
\end{landscape}

%%%%%%%%%%%%%%%%%%%%%%%%%%% Da Modello ER a Modello Relazionale
\chapter{Da Modello ER a Modello Relazionale}
Dopo aver completato lo schema ER, è necessario mappare le entità e le relazioni sul database. Questa procedura avviene secondo i seguenti passi:

\begin{itemize}
\item Traduzione di tipi di entità
\begin{itemize}
\item Traduzione entità forti.
\item Traduzione entità deboli e specializzazioni.
\end{itemize}
\item Traduzione di tipi di associazioni binarie.
\begin{itemize}
\item Traduzioni associazioni 1:1.
\item Traduzione associazioni 1:N.
\item Traduzione associazioni N:M.
\end{itemize} 
\item Traduzione di attributi multivalore.
\item Traduzioni di tipi di associazione N-arie.
\end{itemize}

\section{Traduzione entità forti}
Per ogni tipo di entità forte E nello schema ER, viene costruita una relazione R che contiene tutti gli attributi semplice di E. Di un attributo composto vanno inseriti solamente gli attributi componenti semplici. Come chiave primaria viene scelto uno degli attributi chiavi di E.
Le entità forti presenti nello schema sono:
\begin{itemize}
\item Utente
\item Farmaco
\item Reparto
\item Sala
\item Prestazione
\end{itemize}
Le chiavi primarie sono evidenziate in giallo, le chiavi esterne in verde e gli attributi unici in azzurro. La loro rappresentazione è la seguente:
\begin{figure}[H]
\begin{center}
\includegraphics[scale=0.4]{utenteSchema}
\end{center}
\end{figure}

\begin{figure}[H]
\begin{center}
\includegraphics[scale=0.4]{farmacoSchema}
\end{center}
\end{figure}

\begin{figure}[H]
\begin{center}
\includegraphics[scale=0.4]{repartoSchema}
\end{center}
\end{figure}

\begin{figure}[H]
\begin{center}
\includegraphics[scale=0.4]{salaSchema}
\end{center}
\end{figure}

\begin{figure}[H]
\begin{center}
\includegraphics[scale=0.4]{prestazioneSchema}
\end{center}
\end{figure}

\section{Traduzione entità deboli e specializzazioni}
Per ogni tipo di entità debole W dello scema ER con entità proprietaria E, viene costruita una relazione R che ha come attributi, tutti gli attributi semplici di W. Inoltre vengono inseriti come attributi di chiave esterna in R, le chiavi primarie delle relazioni proprietari. La chiave primaria di R è data dalla combinazione della chiave primaria delle varie entità proprietarie e dell'eventuale chiave parziale dell'entità debole W.
Nello schema è presente un'unica entità debole: \textit{Referto}
\begin{figure}[H]
\begin{center}
\includegraphics[scale=0.4]{refertoSchema}
\end{center}
\end{figure}
Gli attributi \textit{idPrestazione} e \textit{idPaziente} sono rispettivamente le chiavi primarie di \textit{Prestazione} e \textit{Paziente}, entità proprietarie di \textit{Referto}. La chiave primaria di \textit{Referto} è quindi la combinazione di \textit{idPrestazione} e \textit{idPaziente} (\textit{Referto} non ha chiave parziale).\\

Le specializzazioni di \textit{Utente} (\textit{Paziente} e \textit{Staff}) sono rappresentate nel modo seguente:
\begin{figure}[H]
\begin{center}
\includegraphics[scale=0.4]{pazienteSchema}
\end{center}
\end{figure}

\begin{figure}[H]
\begin{center}
\includegraphics[scale=0.4]{staffSchema}
\end{center}
\end{figure}

\section{Traduzioni associazioni 1:1}
L'unica relazione \textit{1:1} è \textit{Emette}, che collega \textit{Prestazione} a \textit{Referto}. Essendo \textit{Referto} una entità debole collegata a \textit{Prestazione}, in \textit{Referto} viene inserita la chiave primaria di \textit{Referto}. 

\section{Traduzioni associazioni 1:N}
Per ogni associazione R binaria del tipo \textit{1:N} nello schema ER, vengono individuate le due relazioni corrispondenti alle due entità partecipanti. Viene quindi inserita nel lato \textit{N}, la chiave primaria della relazione lato \textit{1}. In questo caso sono quindi state inserite le seguenti chiavi:
\begin{itemize}
\item \textit{id} di \textit{Paziente} in \textit{Referto}.
\item \textit{id} di \textbf{Paziente} in \textit{Prestazione}.
\item \textit{id} di \textit{Reparto} in \textit{Staff}.
\item \textit{id} di \textit{Reparto} in \textit{Sala}.
\item \textit{id} di \textit{Sala} in \textit{Prestazione}.
\end{itemize}

\section{Traduzioni associazioni N:M}
Per ogni associazione binaria R del tipo \textit{N:M}, viene costruita una nuova relazione S che la rappresenta. Come attributi di chiave esterna si S, vengono inserite le chiavi primarie delle relazioni rappresentanti le entità partecipanti: la combinazione di queste due chiavi forma la chiave primaria di S. In particolare:

\begin{itemize}
\item StaffPrestazione: contiene, come chiavi esterne, le chiavi primarie delle tabelle \textit{Staff} e \textit{Prestazione}.
\item PazienteFarmaco: contiene, come chiavi esterne, le chiavi primarie delle tabelle \textit{Paziente} e \textit{Farmaco}.
\item PrestazioneFarmaco: contiene, come chiavi esterne, le chiavi primarie delle tabelle \textit{Prestazione} e \textit{Farmaco}.
\end{itemize}

\begin{figure}[H]
\begin{center}
\includegraphics[scale=0.4]{staffPrestazioneSchema}
\end{center}
\end{figure}

\begin{figure}[H]
\begin{center}
\includegraphics[scale=0.4]{pazienteFarmacoSchema}
\end{center}
\end{figure}

\begin{figure}[H]
\begin{center}
\includegraphics[scale=0.4]{farmacoPrestazioneSchema}
\end{center}
\end{figure}


Per ogni associazione \textit{N:M} viene quindi creata una nuova tabella nel database.

%%%%%%%%%%%%%%%%%%%%%%%%%%%%%%%%%%%%%%
\chapter{Normalizzazione}
Il processo di normalizzazione si basa su una serie di test che verificano se uno schema di relazione soddisfa una determinata \textit{forma normale}. Esistono diversi tipi di forma normale:
\begin{itemize}
\item Prima forma normale (1NF).
\item Seconda forma normale (2NF).
\item Terza forma normale (3NF).
\item Forma normale di Boyce e Codd.
\item Quarta forma normale (4NF).
\item Quinta forma normale.
\end{itemize}
L'obiettivo della normalizzazione dei dati è quello di minimizzare la ridondanza e le anomalie dovute all'inserimento, cancellazione o modifica dei dati nel database. Questa caratteristica si ottiene attraverso una analisi degli schemi forniti e una decomposizione degli schemi che non soddisfano certe condizioni, in schemi più piccoli (che verificano le proprietà desiderate). In questa applicazione, si raggiunge solamente la terza forma normale.
\\
La \textit{prima forma normale} richiede che il dominio di un attributo contenga solamente valori indivisibili e che il valore di un qualsiasi attributo in una tupla sia un valore singolo del dominio. Lo schema presentato è già in prima forma normale.
\\
Per soddisfare la \textit{seconda forma normale}, nello schema di relazione R, ogni attributo non primo di R (quindi non fa parte di una chiave candidata) deve dipendete funzionalmente in modo completo dalla chiave primaria di R. La definizione di dipendenza funzionale completa è le seguente: una dipendenza funzionale $X \rightarrow Y$ si dice \textit{dipendenza funzionale completa} se la rimozione di un qualsiasi attributo A da X, comporta che la dipendenza funzionale non sussista più. Per ottenere la seconda forma normale, si decompone la relazione principale R in un certo numero di relazioni nelle quali gli attributi non primi sono associati solamente alla parte della chiave primaria da cui sono funzionalmente dipendenti in modo completo.
\\
Uno schema di relazione R è in terza forma normale se soddisfa la seconda forma normale e nessun attributo non primo di R dipende in modo transitivo dalla chiave primaria.
\begin{figure}
\begin{center}
\includegraphics[scale=0.85]{schema_relazionale}
\end{center}
\end{figure}

%%%%%%%%%%%%%%%%%%%%%%%%%%% NORMALIZZAZIONE


%%%%%%%%%%%%%%%%%%%%%%%%%%% CODICE SQL
\chapter{Codice SQL}
\section{Introduzione}
Come detto nella descrizione, l'intero progetto è stato sviluppato utilizzando il framework \textit{laravel}. Le tabelle del database sono state create utilizzando le \texttt{migration}. Per ogni tabella del database è stato eseguito il comando \texttt{php artisan make:migration create\_table\_nomeTabella}. Questo comando genera una classe \texttt{migration} all'interno del file \texttt{create\_table\_nomeTabella.php} nella quale sono definiti i metodi \texttt{up()} e \texttt{down()}. All'interno di \texttt{up()} vengono inseriti tutti i comandi per la creazione delle tabelle. Generate le migrations per tutte le tabelle, il comando \texttt{php artisan migrate} traduce i comandi specificati nel metodo \texttt{up}, in comandi SQL per la creazione delle tabelle. Di seguito viene riportato il codice SQL generato dalle migrations.
\section{Codice}
\begin{verbatim}
UTENTE
CREATE TABLE utente (
        id INT AUTO_INCREMENT PRIMARY_KEY,
        nome VARCHAR(255) NOT NULL,
        cognome VARCHAR(255) NOT NULL,
        dataNascita DATE NOT NULL,
        sesso BIT NOT NULL,
        codiceFiscale VARCHAR(255) NOT NULL UNIQUE, 
        email VARCHAR(255) NOT NULL UNIQUE,
        password VARCHAR(255) NOT NULL,
        telefono VARCHAR(255) NOT NULL,
        attivo BIT NOT NULL,
        provincia VARCHAR(255) NOT NULL,
        stato VARCHAR(255) NOT NULL,
        comune VARCHAR(255) NOT NULL,
        via VARCHAR(255) NOT NULL,
        numeroCivico INT NOT NULL,
        timestamp TIMESTAMP         
);

PAZIENTE
CREATE TABLE paziente (
        id INT FOREGIN_KEY REFERENCES utente(id),
        note TEXT,
        altezza INT NOT NULL,
        peso INT NOT NULL,
        timestamp TIMESTAMP        
);

REPARTO
CREATE TABLE reparto (
	id INT PRIMARY_KEY,
    nome VARCHAR(255) NOT_NULL, 
    identificativo VARCHAR(255) NOT NULL,
    descrizione TEXT,
    timestamp TIMESTAMP, 
);

SALA
CRATE TABLE sala (
        id INT AUTO INCREMENT PRIMARY KEY,
        identificativo VARCHAR(255) NOT NULL,
        idReparto INT FOREGIN KEY REFERENCES reparto(id),
        piano INT,
       	timestamp TIMESTAMP
);

STAFF
CREATE TABLE staff (
        id INT FOREGIN_KEY REFERENCES utente(id),
        idReparto INT FOREGIN_KEY REFERENCES utente(idReparto),
        identificativo VARCHAR(255) NOT NULL,
        contenuto TETX NOT_NULL,
        timestamp TIMESTAMP
);

FARMACO
CREATE TABLE farmaco (
        id INT AUTO INCREMENT PRIMARY KEY,
        descrizione TEXT NOT NULL,
        nome VARCHAR(255) NOT NULL,
        categoria VARCHAR(255) NOT NULL,
       	timestamp TIMESTAMP
);

PRESTAZIONE
CREATE TABLE prestazione (
        id INT AUTO_INCREMENT PRIMARY_KEY,
        idReparto INT FOREGIN_KEY REFERENCES reparto(idReparto),
        identificativo VARCHAR(255) NOT NULL,
        note TEXT,
        attivo BIT NOT NULL,
        effettuata BIT NOT NULL,
        timestamp TIMESTAMP
);

REFERTO
CREATE TABLE referto (
        idPrestazione INT FOREGIN_KEY REFERENCES prestazione(id),
        idPaziente INT FOREGIN_KEY REFERENCES prestazione(idPaziente),
        identificativo VARCHAR(255) NOT NULL,
        esito TEXT NOT NULL,
        note TEXT,
        timestamp TIMESTAMP
);
\end{verbatim}

%%%%%%%%%%%%%%%%%%%%%%%%%%% INTERROGAZIONI
\chapter{Query}

%%%%%%%%%%%%%%%%%%%%%%%%%%% INTERFACCIA
\chapter{Interfaccia}

%%%%%%%%%%%%%%%%%%%%%%%%% BIBLIOGRAFIA


%%% End document
\end{document}