\documentclass[paper=a4, fontsize=11pt,x11names]{report}
\usepackage[T1]{fontenc}
\usepackage{fourier}

%margine a mano
\usepackage[margin=1.1in]{geometry}

\usepackage[utf8]{inputenc}
\usepackage[italian]{babel}
\usepackage[protrusion=true,expansion=true]{microtype}	
%\usepackage{amsmath,amsfonts,amsthm} % Math packages
\usepackage[pdftex]{graphicx}	
\usepackage{url}
\usepackage{graphicx}


%per il grafico
\usepackage{tikz}
\usetikzlibrary{mindmap,trees}
\usepackage{pgfplots} %per istorgrammi

%%% Custom sectioning
\usepackage{sectsty}
\allsectionsfont{\centering \normalfont\scshape}


%%% Custom headers/footers (fancyhdr package)
\usepackage{fancyhdr}
\pagestyle{fancyplain}
\fancyhead{}											% No page header
\fancyfoot[L]{}											% Empty 
\fancyfoot[C]{}											% Empty
\fancyfoot[R]{\thepage}									% Pagenumbering
\renewcommand{\headrulewidth}{0pt}			% Remove header underlines
\renewcommand{\footrulewidth}{0pt}				% Remove footer underlines
\setlength{\headheight}{13.6pt}

%\setlength{\textwidth}{6.5in}

%%% Equation and float numbering
%\numberwithin{equation}{section}		% Equationnumbering: section.eq#
%\numberwithin{figure}{section}			% Figurenumbering: section.fig#
%\numberwithin{table}{section}				% Tablenumbering: section.tab#




%%%%%%%%%%%%% per ER

\usepackage{tikz} 
\usetikzlibrary{er} 
\usetikzlibrary{arrows,automata}
\usetikzlibrary{positioning}
\tikzset{multi attribute/.style={attribute,double distance=1.5pt}} 
\tikzset{derived attribute/.style={attribute,dashed}} 
\tikzset{total/.style={double distance=1.5pt}} 
\tikzset{every entity/.style={draw=orange, fill=orange!20}} 
\tikzset{every attribute/.style={draw=MediumPurple1, fill=MediumPurple1!20}} 
\tikzset{every relationship/.style={draw=Chartreuse2, fill=Chartreuse2!20}} 
\newcommand{\key}[1]{\underline{#1}}

%%%%%%%%%%%%%%%%%


%%% Maketitle metadata
\newcommand{\horrule}[1]{\rule{\linewidth}{#1}} 	% Horizontal rule


\title{
		%\vspace{-1in} 	
		\usefont{OT1}{bch}{b}{n}
		\normalfont \normalsize \textsc{Universit\`a degli Studi di Ferrara \\ Ingegneria Informatica e dell'Automazione
			\\ Basi di Dati} \\ [25pt]
		\horrule{0.5pt} \\[0.4cm]
		\Huge Realizzazione Database per Social Network \date{}\\%[0.2cm]
		\horrule{0.5pt} \\[0.4cm]
		\LARGE Azzolini Damiano - Bertagnon Alessandro \\ [0.4cm]
		%\normalfont \normalsize Azzolini Damiano, Bertasi Francesco, Decarlo Dario, \\
		%\normalfont \normalsize Fazzi Mattia, Fiorini Giovanni, Fontana Niccol\`o, Rambaldi Marco\\[0.4cm]
		\horrule{2pt} \\[0.8cm]
		\includegraphics{logoUnife}
		%\horrule{2pt} \\[0.5cm]
		%%%%%%%%%%%%%%%%%%%%%%%%%%%%%%% MANCA DATA %%%%%%%%%%%%%%%
}


%%% Begin document
\begin{document}
\maketitle

\newpage

\tableofcontents
\thispagestyle{empty}

\listoffigures
\thispagestyle{empty}


\newpage

\pagenumbering{arabic}
%\begin{abstract}
%Your abstract goes here...
%\end{abstract}

%\chapter{Introduction} %solo per report
%This chapter's content...


%%%%%%%%%%%%%%%%%%%%%%%%%%%%%%%%
%%%%%%%%%%%%%%%           \`u

%%%%%%%%%%%%%%%%%%%%%%%%%%% DESCRIZIONE MINIMONDO
\chapter{Minimondo}
\section{Descrizione}
Il progetto di basa sullo sviluppo di un \textit{social network}. Ciascun utente si può iscrivere al social network
fornendo Nome, Cognome, e-mail e password e data di nascita. Al seguito dell'iscrizione, un utente può chiedere l'"amicizia" ad 
un altro utente, pubblicare un
post, che può anche contenere media, che può essere commentato da altri utenti (solamente quelli tra gli amici dell'utente creatore del post). E possibile inoltre inserire, oltre a commenti, delle reazioni ai post. 
La scrittura di un commento o l'applicazione di una reazione ad un post viene segnalata con una notifica all'utente creatore del post, così come la richiesta di amicizia. Ciascuna entità utente, post, commento è caratterizzata da un campo attivo (flag booleano) che indica se, rispettivamente, l'utente si è cancellato dal social network (in questo caso attivo sarà fissato a 0), il post è stato cancellato o il commento è stato rimosso. I post che vengono cancellati dal social network rimangono comunque salvati nel database ma non visualizzati nella home.

\section{Entità}
Di seguito vengono analizzate tutte le entità presenti nel database:

\subsection{Utente}
\centering
\vspace{0.5cm}
\begin{tikzpicture}[
	%->, frecce orietate
	%>=stealth',
  %	%shorten >=1pt,
  	auto,	
  	semithick, %linee più spesse
	node distance=7em,
	scale=0.8, transform shape] 
	
	\node[entity] (utente) {Utente};
	
	\node[attribute] (nome) [above of = utente] {Nome} edge (utente);
	\node[attribute] (email) [below of = utente] {Email} edge (utente);
	\node[attribute] (cognome) [right of = utente] {Cognome} edge (utente); 
	\node[attribute] (datanascite) [below right of = utente] {DataNascita} edge (utente);	
	\node[attribute] (password) [below left of = utente] {Password} edge (utente);
	\node[attribute] (userid) [left of = utente] {\key{UserID}} edge (utente);
	\node[attribute] (attivo) [above right of =  utente] {Attivo} edge (utente);
	
\end{tikzpicture}
\vspace{0.5cm}

L'entità utente è l'entità principale di tutto il database ed è costituita dai seguenti attributi:
\begin{itemize}
\item Nome: nome dell'utente.
\item Cognome: cognome dell'utente.
\item UserID: identificativo numerico dell'utente e chiave primaria.
\item Password: password dell'account.
\item Email: email dell'utente.
\item DataNascita: data di nascita dell'utente.
\item Attivo: flag booleano, solitamente fissato a 0, 1 se l'utente si è cancellato dal social. 
\end{itemize}

\subsection{Amicizia}

\subsection{Post}
\subsection{Commento}
\subsection{Media}

\subsection{Reazione}
\subsection{Notifica}

\section{Relazioni}

%%%%%%%%%%%%%%%%%%%%%%%%%%% MODELLO ER
\chapter{Modello ER}

%%%%%%%%%%%%%%%%%%%%%%%%%%% CODICE SQL
\chapter{Codice SQL}



%%%%%%%%%%%%%%%%%%%%%%%%% BIBLIOGRAFIA


\begin{thebibliography}{9}


\bibitem{leibold}
 Marius Leibold, Gilbert J. B. Probst, Michael Gibbert,
  \textit{Strategic Management in the Knowledge Economy: New Approaches and Business Applications},
  John Wiley \& Sons,
 2007.
 
\bibitem{cartonChina}
	Tetra Pak launches FSC cartons in China, \textit{http://beta.nepcon.org/newsroom/tetra-pak-launches-fsc-cartons-china},
	10 Giugno 2010.
	
\bibitem{circular}
	Circular Economy, Sustainable Materials Management, and the Importance of KPIs: \textit{https://sustainablepackaging.org/circular-economy-sustainable-materials-management-importance-kpis/},
	17 Maggio 2017.

\end{thebibliography}


%%% End document
\end{document}