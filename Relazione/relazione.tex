\documentclass[paper=a4, fontsize=11pt,x11names]{report}
\usepackage[T1]{fontenc}
\usepackage{fourier}

%margine a mano
\usepackage[margin=1.1in]{geometry}

\usepackage[utf8]{inputenc}
\usepackage[italian]{babel}
\usepackage[protrusion=true,expansion=true]{microtype}	
%\usepackage{amsmath,amsfonts,amsthm} % Math packages
\usepackage[pdftex]{graphicx}	
\usepackage{url}
\usepackage{graphicx}


%per il grafico
\usepackage{tikz}
\usetikzlibrary{mindmap,trees}
\usepackage{pgfplots} %per istorgrammi

%%% Custom sectioning
\usepackage{sectsty}
\allsectionsfont{\centering \normalfont\scshape}

%%% Custom headers/footers (fancyhdr package)
\usepackage{fancyhdr}
\pagestyle{fancyplain}
\fancyhead{}											% No page header
\fancyfoot[L]{}											% Empty 
\fancyfoot[C]{}											% Empty
\fancyfoot[R]{\thepage}									% Pagenumbering
\renewcommand{\headrulewidth}{0pt}			% Remove header underlines
\renewcommand{\footrulewidth}{0pt}				% Remove footer underlines
\setlength{\headheight}{13.6pt}

%\setlength{\textwidth}{6.5in}

%%% Equation and float numbering
%\numberwithin{equation}{section}		% Equationnumbering: section.eq#
%\numberwithin{figure}{section}			% Figurenumbering: section.fig#
%\numberwithin{table}{section}				% Tablenumbering: section.tab#




%%%%%%%%%%%%% per ER

\usepackage{tikz} 
\usetikzlibrary{er} 
\usetikzlibrary{arrows,automata}
\usetikzlibrary{positioning}
\tikzset{multi attribute/.style={attribute,double distance=1.5pt}} 
\tikzset{derived attribute/.style={attribute,dashed}} 
\tikzset{total/.style={double distance=1.5pt}} 
\tikzset{every entity/.style={draw=orange, fill=orange!20}} 
\tikzset{every attribute/.style={draw=MediumPurple1, fill=MediumPurple1!20}} 
\tikzset{every relationship/.style={draw=Chartreuse2, fill=Chartreuse2!20}} 
\newcommand{\key}[1]{\underline{#1}}

%%%%%%%%%%%%%%%%%


%%% Maketitle metadata
\newcommand{\horrule}[1]{\rule{\linewidth}{#1}} 	% Horizontal rule


\title{
		%\vspace{-1in} 	
		\usefont{OT1}{bch}{b}{n}
		\normalfont \normalsize \textsc{Universit\`a degli Studi di Ferrara \\ Ingegneria Informatica e dell'Automazione
			\\ Basi di Dati} \\ [25pt]
		\horrule{0.5pt} \\[0.4cm]
		\Huge Realizzazione Database per Social Network \date{}\\%[0.2cm]
		\horrule{0.5pt} \\[0.4cm]
		\LARGE Azzolini Damiano - Bertagnon Alessandro \\ [0.4cm]
		%\normalfont \normalsize Azzolini Damiano, Bertasi Francesco, Decarlo Dario, \\
		%\normalfont \normalsize Fazzi Mattia, Fiorini Giovanni, Fontana Niccol\`o, Rambaldi Marco\\[0.4cm]
		\horrule{2pt} \\[0.8cm]
		\includegraphics{logoUnife}
		%\horrule{2pt} \\[0.5cm]
		%%%%%%%%%%%%%%%%%%%%%%%%%%%%%%% MANCA DATA %%%%%%%%%%%%%%%
}


%%% Begin document
\begin{document}
\maketitle

\newpage

\tableofcontents
\thispagestyle{empty}

\listoffigures
\thispagestyle{empty}


\newpage

\pagenumbering{arabic}
%\begin{abstract}
%Your abstract goes here...
%\end{abstract}

%\chapter{Introduction} %solo per report
%This chapter's content...


%%%%%%%%%%%%%%%%%%%%%%%%%%%%%%%%
%%%%%%%%%%%%%%%           \`u

%%%%%%%%%%%%%%%%%%%%%%%%%%% DESCRIZIONE MINIMONDO
\chapter{Minimondo}
\section{Descrizione}
Il progetto di basa sulla realizzazione di una applicazione web per la gestione di una clinica privata. Alla piattaforma possono accedere 5 tipi di utente:
\begin{itemize}
\item paziente
\item medico
\item infermiere
\item impiegato
\item amministratore
\end{itemize}  

La clinica in questione eroga diversi tipi di \textbf{prestazioni} ai suoi utenti, ad esempio: visite specialistiche, esami diagnostici, day surgery e terapie. Ogni prestazione può essere effettuata da uno o più membri dello \textbf{staff} (a seconda della complessità) in una delle \textbf{sale} della clinica. Al termine di ogni prestazione il medico compila un \textbf{referto} corrispondete alla prestazione appena effettuata. Il sistema deve anche gestire i \textbf{farmaci} assunti dagli utenti e utilizzati durante le prestazioni. Per motivi di organizzazione interna ogni membro del personale e ogni sala afferisce a uno specifico \textbf{reparto} della clinica. Più in dettaglio:\\

L'utente \textbf{PAZIENTE} potrà:
\begin{itemize}
\item Registrarsi sulla piattaforma e fare il login. Modificare il proprio profilo.
\item Aggiungere/rimuovere i farmaci che assume regolarmente.
\item Visionare le prestazioni effettuate con i referti corrispondenti.
\end{itemize}

L'utente \textbf{MEDICO} potrà:
\begin{itemize}
\item Fare il login sulla piattaforma e visionare il proprio profilo.
\item Visionare le schede personali dei pazienti (compresi i farmaci assunti).
\item Visionare le prestazioni e i relativi referti.
\item Aggiungere/Modificare/Cancellare i referti delle prestazioni a cui ha preso parte.
\item Aggiungere/Rimuovere i farmaci utilizzati nelle prestazioni a cui ha preso parte.
\item Aggiungere personale alle prestazioni che gli sono state assegnate.
\end{itemize}

L'utente \textbf{INFERMIERE} potrà:
\begin{itemize}
\item Fare il login sulla piattaforma e visionare il proprio profilo.
\item Visionare le schede personali dei pazienti (compresi i farmaci assunti).
\item Visionare le prestazioni assegnate.
\item Aggiungere/Rimuovere i farmaci utilizzati nelle prestazioni alle quali ha preso parte.
\end{itemize}

L'utente \textbf{IMPIEGATO} potrà:
\begin{itemize}
\item Fare il login sulla piattaforma e visionare il proprio profilo.
\item Prenotare le prestazioni per i pazienti associando ad esse i medici che dovranno effettuarle.
\item Visualizzare lo storico delle prestazioni effettuate dai pazienti (ma non i referti). 
\item Gestire il personale:
	\begin{itemize}
		\item Modificare lo stipendio dei vari membri dello staff.
		\item Modificare il reparto di appartenenza.
	\end{itemize}
	
\item Aggiungere/Modificare/Cancellare le tipologie di prestazioni. 
\item Aggiungere/Modificare/Cancellare i farmaci nella lista della farmacia.
\end{itemize}

L'utente \textbf{AMMINISTRATORE} potrà:
\begin{itemize}
\item Fare il login sulla piattaforma e visionare il proprio profilo.
\item Fare tutto quello che fanno gli utenti precedenti.
\item Aggiungere/Modificare/Cancellare gli utenti Staff della clinica.
\item Aggiungere/Modificare/Cancellare le sale della clinica.
\item Aggiungere/Modificare/Cancellare i reparti della clinica.
\item Gestire tutta la base utenti.
\item Aggiungere/Rimuovere i singoli ruoli (permessi) agli utenti.
\end{itemize}

\section{Entità}
Di seguito vengono analizzate tutte le entità presenti nel database:


%%%%%%%%%%%%%%%%%%%%%%%%%%%%%%%%%%%%%%
% FINIRE DI MODIFICARE I CAMPI
%%%%%%%%%%%%%%%%%%%%%%%%%%%%%%%%



\subsection{Utente}
\begin{center}
\vspace{0.5cm}
\begin{tikzpicture}[
	%->, frecce orietate
	%>=stealth',
  %	%shorten >=1pt,
  	auto,	
  	semithick, %linee più spesse
	node distance=7em,
	scale=0.8, transform shape] 
	
	\node[entity] (utente) {Utente};
	
	\node[attribute] (id) [above of = utente] {\key{ID}} edge (utente);
	\node[attribute] (nome) [above right of = utente] {Nome} edge (utente);
	\node[attribute] (cognome) [right of = nome] {Cognome} edge (utente);
	\node[attribute] (dataNascita) [below of = utente] {DataNascita} edge (utente);
	\node[attribute] (sesso) [below right of = utente] {Sesso} edge (utente);
	\node[attribute] (cf) [right of = sesso] {C.F.} edge (utente);
	\node[attribute] (email) [below left of = utente] {Email} edge (utente);
	\node[attribute] (pw) [left of = email] {Password} edge (utente);
	\node[attribute] (telefono) [left of = utente] {Telefono} edge (utente);
	\node[attribute] (attivo) [right of = utente] {Attivo} edge (utente);
	
	\node[attribute] (indirizzo) [above left of = utente] {Indirizzo} edge (utente);
	\node[attribute] (stato) [above of = indirizzo] {Stato} edge (indirizzo);
	\node[attribute] (comune) [left of = stato] {Comune} edge (indirizzo);
	\node[attribute] (via) [left of = indirizzo] {Via} edge (indirizzo);
	\node[attribute] (numeroCivico) [right of = stato] {Civico} edge (indirizzo);
	\node[attribute] (provincia) [left of = comune] {Provincia} edge (indirizzo);	
\end{tikzpicture}
\vspace{0.5cm}
\end{center}

\subsection{Paziente}
\begin{center}
\vspace{0.5cm}
\begin{tikzpicture}[
	%->, frecce orietate
	%>=stealth',
  %	%shorten >=1pt,
  	auto,	
  	semithick, %linee più spesse
	node distance=7em,
	scale=0.8, transform shape] 
	
	\node[entity] (paziente) {Paziente};
	
	\node[attribute] (altezza) [above left of = paziente] {Altezza} edge (paziente);
	\node[attribute] (note) [left of = paziente] {Note} edge (utente);
	\node[attribute] (peso) [right of = paziente] {Peso} edge (paziente);
	\node[attribute] (timestamp) [above right of = paziente] {Timestamp} edge (paziente);		
\end{tikzpicture}
\vspace{0.5cm}
\end{center}
L'entità \textit{Paziente} è una specializzazione della entità \textit{Utente}. Ha come chiave esterna l'id dell'utente al quale si riferisce.

\subsection{Staff}
\begin{center}
\vspace{0.5cm}
\begin{tikzpicture}[
	%->, frecce orietate
	%>=stealth',
  %	%shorten >=1pt,
  	auto,	
  	semithick, %linee più spesse
	node distance=7em,
	scale=0.8, transform shape] 
	
	\node[entity] (staff) {Staff};
	
	\node[attribute] (altezza) [above of = staff] {Identificativo} edge (staff);
	\node[attribute] (stipendio) [above left = 1cm of staff] {Stipendio} edge (staff);
		\node[attribute] (timestamp) [above right = 1cm of staff] {Timestamp} edge (staff);		
\end{tikzpicture}
\vspace{0.5cm}
\end{center}
L'entità \textit{Staff}, analogamente ad \textit{Paziente}, è una specializzazione della entità \textit{Utente}. Ha come chiave esterna l'id dell'utente al quale si riferisce e il reparto di appartenenza.

\subsection{Reparto}
\begin{center}
\vspace{0.5cm}
\begin{tikzpicture}[
	%->, frecce orietate
	%>=stealth',
  %	%shorten >=1pt,
  	auto,	
  	semithick, %linee più spesse
	node distance=7em,
	scale=0.8, transform shape] 
	
	\node[entity] (reparto) {Reparto};
	
	\node[attribute] (id) [above of = reparto] {\key{ID}} edge (reparto);
	\node[attribute] (nome) [above left = 1cm of reparto] {Nome} edge (reparto);
	\node[attribute] (identificativo) [above right = 1cm of reparto] {Identificativo} edge (reparto);
	\node[attribute] (descrizione) [left = 1cm of reparto] {Descrizione} edge (reparto);
	\node[attribute] (timestamp) [right = 1cm of reparto] {Timestamp} edge (reparto);				
\end{tikzpicture}
\vspace{0.5cm}
\end{center}

\subsection{Sala}
\begin{center}
\vspace{0.5cm}
\begin{tikzpicture}[
	%->, frecce orietate
	%>=stealth',
  %	%shorten >=1pt,
  	auto,	
  	semithick, %linee più spesse
	node distance=7em,
	scale=0.8, transform shape] 
	
	\node[entity] (sala) {Sala};
	
	\node[attribute] (id) [above of = sala] {\key{ID}} edge (sala);
	\node[attribute] (identificativo) [above left = 1cm of sala] {Identificativo} edge (sala);
	\node[attribute] (descrizione) [above right = 1cm of sala] {Descrizione} edge (sala);
	\node[attribute] (timestamp) [left = 1cm of sala] {Timestamp} edge (sala);	
	\node[attribute] (piano) [right = 1cm of sala] {Piano} edge (sala);				
\end{tikzpicture}
\vspace{0.5cm}
\end{center}
L'entità \textit{Sala} ha come chiave esterna l'id del reparto alla quale è assegnata.

\subsection{Farmaco}
\begin{center}
\vspace{0.5cm}
\begin{tikzpicture}[
	%->, frecce orietate
	%>=stealth',
  %	%shorten >=1pt,
  	auto,	
  	semithick, %linee più spesse
	node distance=7em,
	scale=0.8, transform shape] 
	
	\node[entity] (farmaco) {Farmaco};
	
	\node[attribute] (id) [above of = farmaco] {\key{ID}} edge (farmaco);
	\node[attribute] (nome) [above left = 1cm of sala] {Nome} edge (farmaco);
	\node[attribute] (descrizione) [above right = 1cm of farmaco] {Descrizione} edge (farmaco);
	\node[attribute] (timestamp) [left = 1cm of farmaco] {Timestamp} edge (farmaco);	
	\node[attribute] (categoria) [right = 1cm of farmaco] {Categoria} edge (farmaco);				
\end{tikzpicture}
\vspace{0.5cm}
\end{center}

\subsection{Prestazione}
\begin{center}
\vspace{0.5cm}
\begin{tikzpicture}[
	%->, frecce orietate
	%>=stealth',
  %	%shorten >=1pt,
  	auto,	
  	semithick, %linee più spesse
	node distance=7em,
	scale=0.8, transform shape] 
	
	\node[entity] (prestazione) {Prestazione};
	
	\node[attribute] (id) [below right = 1cm of prestazione] {\key{ID}} edge (prestazione);
	\node[attribute] (effettuata) [above left = 1cm of prestazione] {Effettuata} edge (prestazione);
	\node[attribute] (note) [right = 1cm of prestazione] {Note} edge (prestazione);
	\node[attribute] (attivo) [left = 1cm of prestazione] {Attivo} edge (prestazione);	
	\node[attribute] (timestamp) [above right = 1cm of prestazione] {Timestamp} edge (prestazione);	
	\node[attribute] (identificativo) [below left = 1cm of prestazione] {Identificativo} edge (prestazione);			
\end{tikzpicture}
\vspace{0.5cm}
\end{center}

\subsection{Referto}
\begin{center}
\vspace{0.5cm}
\begin{tikzpicture}[
	%->, frecce orietate
	%>=stealth',
  %	%shorten >=1pt,
  	auto,	
  	semithick, %linee più spesse
	node distance=7em,
	scale=0.8, transform shape] 
	
	\node[entity] (referto) {Referto};
	
	\node[attribute] (note) [above of = referto] {Note} edge (referto);
	\node[attribute] (esito) [above right = 1cm of farmaco] {Esito} edge (referto);
	\node[attribute] (timestamp) [above left = 1cm of referto] {Timestamp} edge (referto);				
\end{tikzpicture}
\vspace{0.5cm}
\end{center}
L'entità \textit{Referto} è una entità debole, collegata a prestazione. Infatti non ha chiave primaria ed ha come chiavi esterne l'id del paziente e l'id della prestazione alla quale fa riferimento.

%\node[isa] (isa1) [below = of student, node %distance=5em] {ISA} edge node [auto,swap] %{disjoint} (student)






\section{Relazioni}
%UNA SUBSECTION PER OGNI RELAZIONE?
%Le relazioni che collegano le entità sono le seguenti:

%genera riceve accetta scrive scatena richiede lancia possiede crea valuta mette contiene

%riceve
\begin{figure}[h!]
\centering
\begin{tikzpicture}[auto, semithick, scale=1, transform shape] 
	
	\node[entity] (notifica) {Notifica};
	\node[relationship] (riceve) [right = 2cm of notifica] {Riceve} edge node {N} (notifica);
	\node[entity] (utente) [right = 2cm of riceve] {Utente} edge node {1} (riceve);
\end{tikzpicture}
\caption{Un utente riceve N notifiche ma una determinata notifica è ricevuta da un solo utente.}
\end{figure}

%richiede
\begin{figure}[h!]
\centering
\begin{tikzpicture}[auto, semithick, scale=1, transform shape] 
	
	\node[entity] (amicizia) {Amicizia};
	\node[relationship] (richiede) [right = 2cm of amicizia] {Richiede} edge node {N} (amicizia);
	\node[entity] (utente) [right = 2cm of riceve] {Utente} edge node {1} (richiede);
\end{tikzpicture}
\caption{Un utente riceve N richieste di amicizia ma una determinata richiesta è ricevuta da un solo utente.}
\end{figure}

%accetta
\begin{figure}[h!]
\centering
\begin{tikzpicture}[auto, semithick, scale=1, transform shape] 
	
	\node[entity] (amicizia) {Amicizia};
	\node[relationship] (accetta) [right = 2cm of amicizia] {Accetta} edge node {N} (amicizia);
	\node[entity] (utente) [right = 2cm of riceve] {Utente} edge node {1} (accetta);
\end{tikzpicture}
\caption{Un utente accetta N richieste di amicizia ma una determinata richiesta è accettata da un solo utente.}
\end{figure}

%scatena
\begin{figure}[h!]
\centering
\begin{tikzpicture}[auto, semithick, scale=1, transform shape] 
	
	\node[entity] (notifica) {Notifica};
	\node[relationship] (scatena) [right = 2cm of notifica] {Scatena} edge node {1} (notifica);
	\node[entity] (Reazione) [right = 2cm of scatena] {Reazione} edge node {1} (scatena);
\end{tikzpicture}
\caption{Una reazione scatena una notifica e una determinata notifica è scatenata da una reazione.}
\end{figure}

%genera
\begin{figure}[h!]
\centering
\begin{tikzpicture}[auto, semithick, scale=1, transform shape] 
	
	\node[entity] (amicizia) {Amicizia};
	\node[relationship] (genera) [right = 2cm of amicizia] {Genera} edge node {N} (amicizia);
	\node[entity] (notifica) [right = 2cm of riceve] {Notifica} edge node {1} (genera);
\end{tikzpicture}
\caption{Una richiesta di amicizia genera una notifica e una determinata notifica e generata da una richiesta.}
\end{figure}

%scrive
\begin{figure}[h!]
\centering
\begin{tikzpicture}[auto, semithick, scale=1, transform shape] 
	
	\node[entity] (utente) {Utente};
	\node[relationship] (scrive) [right = 2cm of utente] {Scrive} edge node {1} (utente);
	\node[entity] (commento) [right = 2cm of riceve] {Commento} edge node {N} (scrive);
\end{tikzpicture}
\caption{Un utente scrive n commenti ma un determinato commento può essere scritto solamente da un utente.}
\end{figure}

%lancia
\begin{figure}[h!]
\centering
\begin{tikzpicture}[auto, semithick, scale=1, transform shape] 
	
	\node[entity] (commento) {Commento};
	\node[relationship] (lancia) [right = 2cm of commento] {Lancia} edge node {1} (commento);
	\node[entity] (notifica) [right = 2cm of lancia] {Notifica} edge node {1} (lancia);
\end{tikzpicture}
\caption{Un commento lancia una notifica e una determinata notifica può essere lanciata da un solo commento.}
\end{figure}

%possiede
\begin{figure}[h!]
\centering
\begin{tikzpicture}[auto, semithick, scale=1, transform shape] 
	
	\node[entity] (commento) {Commento};
	\node[relationship] (possiede) [right = 2cm of commento] {Possiede} edge node {N} (commento);
	\node[entity] (post) [right = 2cm of possiede] {Post} edge node {1} (possiede);
\end{tikzpicture}
\caption{Un determinato commento può essere fatto solamente su un post ma un post può contenere n commenti.}
\end{figure}

%crea
\begin{figure}[h!]
\centering
\begin{tikzpicture}[auto, semithick, scale=1, transform shape] 
	
	\node[entity] (utente) {Utente};
	\node[relationship] (crea) [right = 2cm of utente] {Crea} edge node {1} (utente);
	\node[entity] (post) [right = 2cm of crea] {Post} edge node {N} (crea);
\end{tikzpicture}
\caption{Un utente crea n post ma un determinato post è scritto solamente da un utente.}
\end{figure}

%valuta
\begin{figure}[h!]
\centering
\begin{tikzpicture}[auto, semithick, scale=1, transform shape] 
	
	\node[entity] (reazione) {Reazione};
	\node[relationship] (valuta) [right = 2cm of reazione] {Valuta} edge node {N} (reazione);
	\node[entity] (post) [right = 2cm of crea] {Post} edge node {1} (valuta);
\end{tikzpicture}
\caption{Una reazione valuta un solo post ma un post può essere valutato da più reazioni.}
\end{figure}

%mette
\begin{figure}[h!]
\centering
\begin{tikzpicture}[auto, semithick, scale=1, transform shape] 
	
	\node[entity] (utente) {Utente};
	\node[relationship] (mette) [right = 2cm of utente] {Mette} edge node {1} (utente);
	\node[entity] (reazione) [right = 2cm of mette] {Reazione} edge node {N} (mette);
\end{tikzpicture}
\caption{Un utente può mettere n reazioni ma una reazione può essere messa da un solo utente.}
\end{figure}

%contiene
\begin{figure}[h!]
\centering
\begin{tikzpicture}[auto, semithick, scale=1, transform shape] 
	
	\node[entity] (post) {Post};
	\node[relationship] (contiene) [right = 2cm of post] {Contiene} edge node {1} (post);
	\node[entity] (media) [right = 2cm of contiene] {Media} edge node {N} (contiene);
\end{tikzpicture}
\caption{Un post contiene n media ma un media può essere in un solo post.}
\end{figure}

%eventuali nuovi

\clearpage
\section{Schema ER Completo}
Da inserire in una pagina nuova

\chapter{Normalizzazione}

%%%%%%%%%%%%%%%%%%%%%%%%%%% Da Modello ER a Modello Relazionale
\chapter{Da Modello ER a Modello Relazionale}

%%%%%%%%%%%%%%%%%%%%%%%%%%% NORMALIZZAZIONE


%%%%%%%%%%%%%%%%%%%%%%%%%%% CODICE SQL
\chapter{Codice SQL}
\section{Introduzione}
Come detto nella descrizione, l'intero progetto è stato sviluppato utilizzando il framework \textit{laravel}. Le tabelle del database sono state create utilizzando le \texttt{migration}. Per ogni tabella del database è stato eseguito il comando \texttt{php artisan make:migration create\_table\_nomeTabella}. Questo comando genera una classe \texttt{migration} all'interno del file \texttt{create\_table\_nomeTabella.php} nella quale sono definiti i metodi \texttt{up()} e \texttt{down()}. All'interno di \texttt{up()} vengono inseriti tutti i comandi per la creazione delle tabelle. Generate le migrations per tutte le tabelle, il comando \texttt{php artisan migrate} traduce i comandi specificati nel metodo \texttt{up}, in comandi SQL per la creazione delle tabelle. Di seguito viene riportato il codice SQL generato dalle migrations.
\section{Codice}
\begin{verbatim}
UTENTE
CREATE TABLE utente (
        utenteID INT AUTO_INCREMENT PRIMARY_KEY,
        nome VARCHAR(255) NOT NULL,
        cognome VARCHAR(255) NOT NULL,
        email VARCHAR(255) NOT NULL UNIQUE,
        password VARCHAR(255) NOT NULL,
        citta VARCHAR(255) NOT NULL,
        dataNascita DATE
);

NOTIFICA
CREATE TABLE notifica (
        notificaID INT AUTO_INCREMENT PRIMARY_KEY,
        utenteID VARCHAR(255) FOREGIN_KEY REFERENCES utente(utenteID),
        tipo VARCHAR(255),
        tipoID VARCHAR(255),
        letta BIT DEFAULT 0
);

AMICIZIA
CREATE TABLE amicizia (
        utenteID1 INT FOREGIN KEY REFERENCES utente(utenteID),
        utenteID2 INT FOREGIN_KEY REFERENCES utente(utenteID),
        timestamp TIMESTAMP,
        Stato VARCHAR(255) NOT_NULL DEFAULT 'sospesa'
);

COMMENTO
CRATE TABLE commento (
        commentoID INT AUTO_INCREMENT PRIMARY_KEY,
        utenteID INT FOREGIN_KEY REFERENCES utente(utenteID),
        postID INT FOREGIN_KEY REFERENCES post(postID),
        contenuto TEXT NOT_NULL,
       	timestamp TIMESTAMP
);

POST
CREATE TABLE post (
        postID INT AUTO_INCREMENT PRIMARY_KEY,
        contenuto TETX NOT_NULL,
        timestamp TIMESTAMP,
        utenteID FOREGIN_KEY REFERENCES utente(utenteID)
);

MEDIA
CREATE TABLE media (
        mediaID INT AUTO_INCREMENT PRIMARY_KEY,
        postID INT FOREGIN_KEY REFERENCES post(PostID),
        percorso TEXT NOT_NULL
);

REAZIONE
CREATE TABLE rezione (
        reazioneID INT AUTO_INCREMENT PRIMARY_KEY,
        utenteID INT FOREGIN_KEY REFERENCES utente(utenteID),
        postID INT FOREGIN_KEY REFERENCES post(postID),
        flag BIT
);
\end{verbatim}

%%%%%%%%%%%%%%%%%%%%%%%%%%% INTERROGAZIONI
\chapter{Query}

%%%%%%%%%%%%%%%%%%%%%%%%%%% INTERFACCIA
\chapter{Interfaccia}

%%%%%%%%%%%%%%%%%%%%%%%%% BIBLIOGRAFIA


\begin{thebibliography}{9}

 
\bibitem{cartonChina}
	Tetra Pak launches FSC cartons in China, \textit{http://beta.nepcon.org/newsroom/tetra-pak-launches-fsc-cartons-china},
	10 Giugno 2010.
	
\bibitem{circular}
	Circular Economy, Sustainable Materials Management, and the Importance of KPIs: \textit{https://sustainablepackaging.org/circular-economy-sustainable-materials-management-importance-kpis/},
	17 Maggio 2017.


\bibitem{leibold}
 Marius Leibold, Gilbert J. B. Probst, Michael Gibbert,
  \textit{Strategic Management in the Knowledge Economy: New Approaches and Business Applications},
  John Wiley \& Sons,
 2007.

\end{thebibliography}


%%% End document
\end{document}